%!TEX TS-program = xelatex
%!TEX encoding = UTF-8 Unicode
% Awesome CV LaTeX Template for Cover Letter
%
% This template has been downloaded from:
% https://github.com/posquit0/Awesome-CV
%
% Authors:
% Claud D. Park <posquit0.bj@gmail.com>
% Lars Richter <mail@ayeks.de>
%
% Template license:
% CC BY-SA 4.0 (https://creativecommons.org/licenses/by-sa/4.0/)
%


%-------------------------------------------------------------------------------
% CONFIGURATIONS
%-------------------------------------------------------------------------------
% A4 paper size by default, use 'letterpaper' for US letter
\documentclass[11pt, a4paper]{awesome-cv}

% Configure page margins with geometry
\geometry{left=1.4cm, top=.8cm, right=1.4cm, bottom=1.8cm, footskip=.5cm}

% Specify the location of the included fonts
\fontdir[fonts/]

% Color for highlights
% Awesome Colors: awesome-emerald, awesome-skyblue, awesome-red, awesome-pink, awesome-orange
%                 awesome-nephritis, awesome-concrete, awesome-darknight
\colorlet{awesome}{awesome-emerald}
% Uncomment if you would like to specify your own color
% \definecolor{awesome}{HTML}{CA63A8}

% Colors for text
% Uncomment if you would like to specify your own color
% \definecolor{darktext}{HTML}{414141}
% \definecolor{text}{HTML}{333333}
% \definecolor{graytext}{HTML}{5D5D5D}
% \definecolor{lighttext}{HTML}{999999}

% Set false if you don't want to highlight section with awesome color
\setbool{acvSectionColorHighlight}{true}

% If you would like to change the social information separator from a pipe (|) to something else
\renewcommand{\acvHeaderSocialSep}{\quad\textbar\quad}


%-------------------------------------------------------------------------------
%	PERSONAL INFORMATION
%	Comment any of the lines below if they are not required
%-------------------------------------------------------------------------------
% Available options: circle|rectangle,edge/noedge,left/right
\photo[circle,noedge,left]{./examples/profile}
\name{Christophe}{Van Neste}
\position{Postdoctoral Researcher{\enskip\cdotp\enskip}Bioinformatician{\enskip\cdotp\enskip}Software Engineer}
\address{Violierstraat 57, 8000 Brugge, BELGIUM}

\mobile{(+32) 486-57-57-51}
\email{Christophe@Van-Neste.be}
\homepage{www.dicaso.be}
\github{beukueb}
\linkedin{cvanneste}
% \gitlab{gitlab-id}
% \stackoverflow{SO-id}{SO-name}
% \twitter{@twit}
% \skype{skype-id}
% \reddit{reddit-id}
% \extrainfo{extra informations}

\quote{``Philosopher and [bio]engineer in theory, visionary [bio]programmer in practice."}


%-------------------------------------------------------------------------------
%	LETTER INFORMATION
%	All of the below lines must be filled out
%-------------------------------------------------------------------------------
% The company being applied to
\recipient
  {Science Division}
  {Yale-NUS College, Singapore}
% The date on the letter, default is the date of compilation
\letterdate{\today}
% The title of the letter
\lettertitle{Job Application for Assistant Professor in Life Sciences}
% How the letter is opened
\letteropening{Dear Mr./Ms./Dr.,}
% How the letter is closed
\letterclosing{Sincerely,}
% Any enclosures with the letter
\letterenclosure[Attached]{Curriculum Vitae}


%-------------------------------------------------------------------------------
\begin{document}

% Print the header with above personal informations
% Give optional argument to change alignment(C: center, L: left, R: right)
\makecvheader[R]

% Print the footer with 3 arguments(<left>, <center>, <right>)
% Leave any of these blank if they are not needed
\makecvfooter
  {\today}
  {Christophe Van Neste~~~·~~~Cover Letter}
  {}

% Print the title with above letter informations
\makelettertitle

%-------------------------------------------------------------------------------
%	LETTER CONTENT
%-------------------------------------------------------------------------------
\begin{cvletter}

  \lettersection{About Me} In 2008, I obtained a Master degree in
  Bio-Engineering at Ghent University (Ghent, Belgium). In the
  following years I studied philosophy and in 2010 obtained a Master
  degree in Philosophical Sciences at Ca’Foscari University (Venice,
  Italy). Thereafter, I became a full-time doctoral fellow in the lab
  of Prof. Dieter Deforce (Ghent University) where I worked as a
  bioinformatician and endeavored on a project to apply massively
  parallel sequencing to forensic DNA profiling analyses. During March
  and May 2014, I had a 6 weeks’ internship at the Illumina
  headquarters in San Diego (USA), where I had the opportunity of
  working closely with the Illumina BaseSpace development team and
  Illumina’s forensic team. I successfully defended my doctoral
  dissertation “Porting forensic DNA analysis to deep sequencing” for
  an international committee in June 2015. Shortly thereafter, I fully
  committed to cancer research and joined the team of Prof. Frank
  Speleman of the Ghent Center for Medical Genetics to work on
  replicative stress in neuroblastoma. In 2016 I won an FWO supported
  research mandate for three years to work on “Computational probing
  of replicative stress resistance and induced G-quadruplex resolving
  processes in embryonic stem or cancer cells.” Since July, I also
  joined the knowledge mining lab of Prof. Vladimir Bajic (King
  Abdullah University of Science and Technology) to work on text
  mining and machine learning in life science research.

  \lettersection{Why Yale-NUS College?}  I recently visited Singapore,
  and out of curiosity I looked up if there were any assistent
  professor positions for computational biology in the area, landing
  me on the present job advertisement. As shared above, my own
  background is quite broad, including bio-engineering, pharmaceutical
  sciences, and philosophy. Giving this context I am intrigued by the
  liberal arts system, and interested to know if I would be a good
  fit.

  \lettersection{Why Me?} I have been working and researching in the
  field of bioinformatics since 2011, including three years and a half
  of postdoctoral experience. In effect, I always assumed going for a
  job in industry after 2 or 3 postdoc years. This year however I have
  had a transformative experience: my principal investigator, Vladimir
  Bajic, involved me in his course "AI in bioinformatics" where I
  introduced computer science students to bioinformation
  theory. Although I was quite struck by how little biological
  knowledge these students had, I very much enjoyed teaching them on
  the subject and finding ways to make the subject more accessbible to
  them giving their computational background. I also enjoyed
  challenging them to become more critical thinkers. This experience
  opened up for me a desire to pursue a career as a professor.  During
  the last years, I have also developed a set of personal insights in
  cancer biology that help me formulating novel hypotheses and topics
  for research, on which I would like master or PhD students to work
  on. Although I have not been educated in the liberal arts system,
  giving my broad background I believe to be well suited for the
  position. I would much appreciate a chance at a sample class to
  prove my worth as far as my teaching qualities are concernced. For
  my research qualities, my publications should sufficiently inform
  you.

\end{cvletter}


%-------------------------------------------------------------------------------
% Print the signature and enclosures with above letter informations
\makeletterclosing

\end{document}
